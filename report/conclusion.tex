\section{Conclusion}
\label{sec:conclusion}

In this project, we successfully completed a complete model of the Algorand consensus protocol in Coq. In doing so, we mechanically verified the asynchronous safety theorem, a key claim of the Algorand protocol, while also outlining the precise assumptions under which the safety theorem holds. Proving this (along with the vast majority of lemmas proved leading to asynchronous safety) gives assurance to the security of the Algorand protocol, which is extremely important for blockchain protocols with potentially billions of dollars at stake.

Concerning future efforts, the model we developed as part of this effort is generic, in that it captures the dynamics of the Algorand consensus protocol in a way that is orthogonal to the properties that we verify about it. This means that the model can be readily used to verify other properties of the system beyond asynchronous safety, including most importantly liveness. In fact, we anticipate that many of the smaller results shown about the protocol and used in the proof of safety will also constitute essential ingredients of the liveness proving effort. Nevertheless, proving liveness will probably require showing additional results, especially those related to timely message delivery and network partitioning, some of which were not needed for the safety argument. But the model already has these components (time, message delays and network partitioning) and going into investigating liveness (and perhaps other properties) will be a seamless continuation of the effort involved in this project.
