\section{The Safety Theorem}
\label{sec:safety}

The safety theorem says that only one block may be certified in a round,
which means the blockchain will not fork.
The key assumptions of the Coq proof are some properties of cryptographic self-selection.
The cryptographic argument for the correctness of cryptographic self-selection
is given in \cite{}, but we have not formalized cryptographic reasoning so
our development takes these properties as unproved assumptions.

The statement of the theorem says that any two certificates from the same round
must be for the same block:

\begin{lstlisting}[language=Coq]
Theorem safety : forall (g0:GState) (trace:seq GState) (r:nat),
  state_before_round r g0 ->
  is_trace g0 trace ->
  forall (p1:nat) (v1:Value), certified_in_period trace r p1 v1 ->
  forall (p2:nat) (v2:Value), certified_in_period trace r p2 v2 ->
  v1 = v2.
\end{lstlisting}


Note the preconditions \lstinline{state_before_round r g0}, which states that
that initial state \lstinline{g0} is far enough back in history that no user
had yet taken any actions in round r, and \lstinline[breaklines=true]{is_trace g0 trace},
which states that the trace is a valid execution trace beginning at g0,
with all steps taken according to the global transition relation.

It is possible to end up with certificates from multiple periods of a round,
during a "network partition" which allowing the adversary to delay messages,
if cert-vote messages are delayed enough for some nodes to advance past the
period where the first certificate was produced, but these multiple certificates
will still be for the same block.

The proof of the main safety theorem[LINK safety] stated above first considers two
cases, for whether the two certificates are from the same period or different
periods.

The case when the certificates are in the same period is handled in the lemma
\lstinline{one_certificate_per_period}.
This proof uses the quorum hypotheses to conclude that there is an honest node
that contributed a cert-vote to both certificates,
and concludes by the lemma \lstinline{no_two_certvotes_in_p} which shows that an
honest node cert-votes at most once in a period (this is proved by analyzing
all transition steps an honest node can take).

The case of different periods is proved using an invariant, which is established
in the period that produces the first certificate and keeps holding for all later
periods of the same round.
The invariant property is that no step of the period produced a quorum of
open next-votes, and any quorum of value next-votes must be for a given value $v$.
This property is defined as \lstinline{period_nextvoted_exclusively_for}.

Lemma \lstinline{certificate_is_start_of_next_period} shows that this is true of
next votes from the period that produced the first certificate,
Lemma \lstinline{excl_enter_excl_next} shows that if this is true for one
period, it is true for the next.
Lemma \lstinline{prev_period_nextvotes_limits_cert_vote} shows that if the
invariant holds for one period then any certificate produced in the next period
can only be for the same value.

Most of the proofs only need the simpler quorum hypotheses that say that any
two quorums from a single step have an honest user in common.
A more complicated quorum hypothesis which relates different steps
is only needed for proving the lemma
\lstinline{certificate_is_start_of_next_period}.
The intuition is that if committee members are selected approximately randomly from
the full population of users, then any property which is true for all honest
members of one quorum must be true for most honest users in the overall population,
so any other quorum even at different steps must have at least some honest
users that also have that property.
But this isn't true if the property is that of being a committee member on
a particular step, and we don't know how to formalize the idea of a property
being unrelated to committee membership, so we assume this hypothesis only
for the specific predicate we need.

The inter-quorum hypothesis is used in between the cert-voting step and
that produced the first certificate any later next-voting steps in the same period,
with the property being whether a user had seen a quorum of soft-votes by the
time that user finished the cert-voting step.
By definition of a certificate and analysis of the transitions which send
cert-votes, this property holds for every honest user which contributed a cert-vote
to the certificate.
Then by the inter-quorum assumptions, any quorum of next-votes must include
a vote from at least one honest user which had seen a quorum of soft-votes before
making their next-vote.
By examining possible transitions this user couldn't make an open next-vote and
could only next-vote for the value which received a certificate, so the invariant
is established.

To prove the latter two lemmas \lstinline{excl_enter_excl_next} and
\lstinline{prev_period_nextvotes_limits_cert_vote} we consider that any quorum
of next-votes or cert-votes contains an honest voter, and looking at the
preconditions for an honest user to make a cert-vote or next-vote.
In either case making such a vote for a value besides $v$ or making an open
next-vote requires that the honest user had received a quorum of
soft-votes for another value or a quorum of open next-votes from the
previous period.
The latter is ruled out because the invariant says no such quorum of votes
was sent, and lemma \lstinline{no_bottom_quorums_during_from_nextvoted_excl}
contains the argument that therefore no such quorum of votes could
be received.
Lemma \lstinline{no_bottom_quorums_during_from_nextvoted_excl}
states that any quorum of soft-votes must be votes for value $v$ if the
invariant for value $v$ held in the previous period.
It is proved by analyzing the transitions which move a user to a new period
to show that all honest users get $v$ as their ``starting value'',
and analyzing the transitions which send soft-votes to show that this
starting value and the absence of quorums of open next-votes means any honest
soft-vote can only be for value $v$.

By the overall inductive argument, we then have the safety theorem that the
protocol cannot fork.